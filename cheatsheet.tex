\documentclass[a4paper,12pt]{article}
\usepackage[utf8]{inputenc}
\usepackage[english]{babel}
\usepackage[T1]{fontenc}
\usepackage[a4paper]{geometry}
\geometry{hscale=0.70,vscale=0.70,centering}
\setlength{\parindent}{0pt}
\setlength{\parskip}{\medskipamount}

\begin{document}

\section{Definition}

\begin{description}
    \item [Pulp] A soft matter (for instance the brain pulp in the car when they
    shot Marvin).
    \item [Pulp magazine] Popular crime novels, with graphic violence and punchy
    dialogues.
\end{description}

\section{Story (sequencial)}

Non-sequencial narration of three different stories:
\begin{itemize}
    \item Vincent Vega
    \item Jules Winnfield
    \item Butch Coolidge
\end{itemize}

\paragraph{Why?} Audience gets more engaged in the movie. Harder to guess how
the different stories will intersect. Deeper understanding of the correlation of events.

% Taken from http://wiki.tarantino.info/index.php/Pulp_Fiction_chronological

\subsection*{Day 1}

Vincent Vega and Jules Winnfield are in the car discussing "carpool" talk (e.g.
quarter pounder in Paris is called a Royale). They also talk about the assignement
of Vincent: to escort Mia, the wife of Marcellus (their boss). They are going to
an apartment to retrieve Marcellus' attache case and to kill the occupants. There
is a third man hiding in the bathroom. He bursts out shooting and miraculously
misses hitting Jules and Vincent. Jules takes this as a miracle.

Vincent and Jules take Marvin (their man inside) into their car to go to Marcellus
Wallace's bar, but Vincent accidentally blows off Marvin's head, causing a very
large bloody mess in the car. Jules calls his old pal Jimmie, and they drive there
to find a way to get out of the unfortunate mess Vincent has gotten them into.
Marcellus sends Wolf to Jimmie's house. Jules and Vincent clean up the car, change
into some ridiculous shorts and t-shirts provided by
Jimmy after they have cleaned off the blood and brain pulp from their bodies.
They follow The Wolf to some car scrap yard and the car and the body are
safely disposed of.

\subsection*{Day 1 (later that morning)}

Jules and Vincent go to a diner to eat before going to Marcellus' bar. Jules
continues his discussion about the miracle, and that he is going to be leaving
the life of gangster. Vincent interrupts to go to the bathroom. Pumpkin and Honey
Bunny decide to rob the dinner and hold everyone at bay with guns and threats of
murder. They collect wallets, and are confronted by Jules when he refuses to give
up the briefcase. Through intimidation, Jules gets back his bad mother fucker wallet,
but gives the robbers his money. He does not kill them, he explains, because of
this recent miracle he experienced.

\subsection*{Day 1 (later that morning continued)}

Vincent and Jules go to Marcellus' bar to deliver the briefcase. They have to wait
because Marcellus is talking to a boxer named Butch. Butch is being paid off to
purposely lose a fight the next night.

\subsection*{Day 1 (early evening)}

Vincent goes to his drug dealer's house and buys heroin. He shoots up, and then
leaves for his evening with Mia. Mia awaits Vincent with surveillance cameras.
While he waits for her, she snorts coke and then makes her appearance.

They go a bizarre retro 50's club. They eat, talk, and enter a dance contest,
which they win. Back at Mia's house, Vincent goes to the bathroom to try and figure
out how he can leave without getting into trouble. While he is in the bathroom,
Mia sings and dances around the room. She finds the heroin in Vincent's pocket,
and thinking it is coke, snorts. She does an overdose and Vincent rushes her to
his drug dealer Lance's house in hopes of getting help. With a giant hypodermic
needle full of adrenaline, Mia is revived. Vincent takes her home, and they both
agree that Marcellus never need know of the events of the evening.

\subsection*{Day 2 (early evening)}

Flashback sequence to the story of Butch and his father's watch and its travels
as told by Butch's father's friend and fellow POW, Captain Koons. We return to
the present, and Butch jumps up, ready for his fight. However, instead of throwing
the fight, he knocks out his opponent, and kills him. He jumps into a cab and goes
to the motel where his girlfriend is waiting. Marcellus is furious, and orders
Butch found and killed.

\subsection*{Day 3 (early in the morning)}

In the morning, Butch realizes that his girlfriend forgot his watch back in his
apartment. Butch is furious, and decides to go back to his apartment for his watch.

Marcellus has put out a hit, and dispatched Vincent to wait for Butch should he
return to his apartment. Butch does return, but as Vincent is once again in the
bathroom, and is able not only to retrieve his watch, but to kill Vincent as well.
Vincent's story is done.

Butch feels lucky. He drives away with a smile and literally bangs into Marcellus
on the street carrying burgers and cokes. There is a bloody confrontation, and
Marcellus chases Butch into a sleazy gun shop. The owner Maynard knocks Butch out
with the butt of his shotgun, then calls his friend Zed, and then brings them both
down to the basement,which we quickly find out is an S/M playground. Marcellus is
chosen to be the victim first, and Butch is able to untie himself and flee. On the
way out, he decides to go back and save Marcellus. He chooses a Samurai sword.
Marcellus is being raped by one of the men, and Butch saves him by killing the
other guy. Marcellus is appropriately thankful and decides that whatever existed
between he and Butch before, it is now a dead issue. Butch is free to go, providing
he never mention this to anyone, or come back to L.A. again.

Butch rides off on Zed's chopper, and returns to the motel where Fabienne is waiting.
They both ride off happily into the sunset, heading to Knoxville. Butch is the
winner in a story of losers.

\section{Filmography}

Several scenarios written before, but no money to do the movies.

\begin{description}
    \item[Reservoir Dogs (1992)] First movie, with gangsters and violence.
    \item[Pulp Fiction (1994)] Again a gangster movie with a lot of violence and humor. Huge success.
    \item[Jackie Brown (1997)] Adaptation from a book. Less successfull.
    \item[Kill Bill: Volume 1 (2003) and 2 (2004)] Had the idea of this movie during the shot of Pulp Fiction. Inspired from Western and Kung Fu movies. A great success.
    \item[Death Proof (2007)] A movie about a psychopathic man. Less successfull.
    \item[Inglourious Basterds (2009)] A war movie, with also a lot of violence and humor. Great success.
    \item[Django Unchained (2012)] A western. Great success.
\end{description}

\section{Actors}

\begin{description}
    \item[John Travolta as Vincent Vega] Tarantino wanted Michael Madsen for this
role, but he was not available. Travolta was already famous, but this revitalized
his career. People think that he made the famous twisting scene incredible because
he was known to have dancing role and he was part of american culture.
    \item[Samuel L. Jackson as Jules Winnfield] This role has been written with
Jackson in mind. This role contributed to start his career.
    \item[Bruce Willis as Butch Coolidge] He was a major star. Taking a role in
a low-budget film meant lowering his salary and risking his star status, but the
strategy was finally a good choice for him.
    \item[Ving Rhames as Marsellus Wallace] Sid Haig was firstly chosen for this
role.
    \item[Uma Thurman as Mia Wallace] This role really started her career. She
apears again in Tarantino's movies: Kill Bill.
    \item[Harvey Keitel as Winston Wolf] This role has been written specifically
for him. He already played in Reservoir Dogs.
\end{description}

\section{Analysis 1 : Breakfast}

Long shot to describe the piece when Vincent and Jules enter.
Low-angle shots for Vincent and Jules symbolizing their superpower whereas there are POV shots (from Vincent/Jules) to film the youths from above, sat.
Bible quoting : for Jules it justifies his acts, for us it just emphasizes the dramatic act and the assurance that Jules has, as a cold-blood hatchet man (homme de main). 

\section{Analysis 2 : Marvin is shot dead}

This scene shows the recurrent mix of violence and humor in Tarantino's movies.
Filmed from behind the shoulders of Vincent and Jules, making us forget that Marvin is sat at the back of the car $\leftarrow$ the shot is more surprising but the result seems insignificant (just like Marvin is in the film). In a car : carpool discussion again AND carpool acts, meaningless.
Absurdism and loss of credibility for Jules and Vincent : they are from the beginning of the movie bad boys and cold-blood killers. Here they just don't know what to do of the dead body of Marvin, they need Wolf which is the brain of this operation but does not kill anybody. At the end they have ridiculous wears to emphasize this.

\end{document}
